\documentclass[10pt]{article}
%\input{page}
\usepackage{fullpage}	% Does about the same as the \input{page}...
% ^ part of the preprint package...

\input{symbolmac}
\usepackage{amsmath}
\usepackage{amssymb}  	% provides Rbb, etc.
\usepackage{latexsym}  	% provides \Box
%\usepackage{theorem} \input{proof}
%\pagestyle{empty}
\usepackage{graphicx} 	% for inserting graphics.
%\usepackage{booktabs}	% for fancy tabular environments and stuff
%\usepackage{subfig}	% for subfigures... 
%(This breaks Beamer if the patch isn't applied!!)
%usepackage{hyperref}


%% ******** For code: ******************
%\usepackage{listings}
%\usepackage[usenames,dvipsnames]{color}
%% ******** Define the language: *******
%\lstdefinelanguage{matlabfloz}{%
%	alsoletter={...},%
%	morekeywords={% 						% keywords
%	break,case,catch,continue,elseif,else,end,for,function,global,%
%	if,otherwise,persistent,return,switch,try,while,...,
%        classdef,properties,methods},%
%	comment=[l]\%,% 						% comments
%	morecomment=[l]...,% 					% comments
%	morestring=[m]',%   						% strings 
%}[keywords,comments,strings]%
%\lstdefinestyle{matlab}{language=matlabfloz,							
%		keywordstyle=\color[rgb]{0,0,1},					
%		commentstyle=\color[rgb]{0.133,0.545,0.133},	% comments
%		stringstyle=\color[rgb]{0.627,0.126,0.941},	% strings
%		numbersep=3mm, numbers=left, numberstyle=\tiny% number style
%}
%% ******** Set the style: **************
%\lstset{style=matlab}

%\newcommand{\matlab}{{\sc Matlab }} % Or 
\renewcommand{\matlab}{{\sc Matlab }} 

% \newcommand{\homedir}{/home/aepound}
% \newcommand{\figdir}{\homedir/figs}  % picture stuff: ps,fig,tex
%\newtheorem{theorem}{Theorem}
%\newtheorem{corollary}{Corollary}
%\newtheorem{lemma}{Lemma}

\title{}
\author{}
\date{Aug. 28, 2014}



\begin{document}
\maketitle

%\begin{abstract}
%\end{abstract}

\section{State Space Representation}
\begin{align}
  \xdot(t) &= A(t) x(t) + B(t)u(t)\\
  y(t)     &= C(t) x(t) + D(t)u(t)
\end{align}

\subsection{Inverted Pendulum on a Cart}

%%%%%%%%%%%%%%%%%%%%%%%%%%%%%%
Insert the diagram here...
%%%%%%%%%%%%%%%%%%%%%%%%%%%%%%

We can define our coordinates as 
\begin{equation}
  q =
  \begin{bmatrix}
    z\\ \theta
  \end{bmatrix}.
\end{equation}
We will be doing this using the Lagrangian Method, so we need to find
out Kinetic and Potential Energies.
The Kinetic Energy ($T$) is
\begin{equation}
  \begin{split}
  T & = T_{cart} + T_{ball}\\
  T_{cart} & = \frac{1}{2} M\zdot^2\\
  T_{ball} & = \frac{1}{2} m\zdot_1^2 + \frac{1}{2} m\zdot_2^2.
  % Why not use angular velocity?
  \end{split}
\end{equation}
Finding what each of the terms in these equations we get
\begin{align}
  z_1 &= z + l\sin(\theta) & z_2 &= l\cos(\theta)\\
  \zdot_1 &= \zdot + l\cos(\theta)\thetadot & \zdot_2 &=
  -l\sin(\theta)\thetadot \\
  \zdot_1^2 &= \zdot^2 + l^2 \cos^2(\theta)\thetadot^2 + 2\zdot l
  \cos(\theta)\thetadot & \zdot_2^2 &= l^2\sin^2(\theta)\thetadot^2.
\end{align}
Thus, we get 
\begin{equation}
  T_{ball} = \frac{1}{2}m\left(\zdot^2 + l^2\thetadot^2 +
    2l\cos(theta)\thetadot \right).
\end{equation}
The total Kinetic Energy of the system is then
\begin{equation}
  T = \frac{1}{2}\zdot^2 + \frac{1}{2}ml^2\thetadot^2 +
  ml\zdot\cos(\theta)\thetadot.
\end{equation}
The Potential Energy of the system is 
\begin{equation}
  V = mgl\cos(\theta).
\end{equation}
Putting these together, we get the full Lagrangian
\begin{equation}
  L = \frac{1}{2} (M + m)\zdot^2 + \frac{1}{2}ml^2\thetadot^2 +
  ml\zdot\cos(\theta)\thetadot - mgl\cos(\theta).
\end{equation}

Now, we need to compute the following:
\begin{equation}
  \begin{split}
    \frac{d}{dt}\left(\partiald{L}{\zdot}\right) -\partiald{L}{z} &=
    F\\
    \frac{d}{dt}\left(\partiald{L}{\thetadot}\right)
    - \partiald{L}{\theta} & = 0. % no external force on the ball..
  \end{split}
\end{equation}
Breaking this down, we get
\begin{equation}
  \begin{split}
    \partiald{L}{\zdot} &= (M+m)\zdot + ml\cos(\theta)\thetadot\\
    \partiald{L}{z} &= 0\\
    \partiald{L}{\thetadot} &= \frac{1}{2}ml^2\thetadot +
    ml\zdot\cos(\theta)\\ 
    \partiald{L}{\theta} &= -ml\zdot\sin(\theta)\thetadot +
    mgl\sin(\theta).\\ 
  \end{split}
\end{equation}
Now for the derivatives with respect to time,
\begin{equation}
  \frac{d}{dt}\left((M+m)\zdot + ml\cos(\theta)\thetadot\right) =
  (M+m)\zddot + ml\cos(\theta)\thetaddot. 
\end{equation}
This gives rise to the equation
\begin{equation}
 (M+m)\zddot + ml\cos(\theta)\thetaddot  - ml\thetadot^2\sin(\theta) = F.
\end{equation}
Do the same for the other equation, and you obtain
\begin{equation}
  ml\cos(\theta)\zddot + ml^2\thetaddot - mgl\sin(\theta) = 0.
\end{equation}
Sticking this into a vector format we obtain the equation
\begin{equation}
  \begin{bmatrix}
    M+m & ml\cos(\theta)\\ ml\cos(\theta) & ml^2
  \end{bmatrix}
  \begin{bmatrix}
    \zddot\\ \thetaddot
  \end{bmatrix}
  -
  \begin{bmatrix}
    ml\thetadot^2\sin(\theta)\\ mgl\sin(\theta)
  \end{bmatrix}
  =
  \begin{bmatrix}
    F \\ 0
  \end{bmatrix}.
\end{equation}
Solving for the state equations yields
\begin{equation}
  \begin{bmatrix}
    \zddot\\ \thetaddot
  \end{bmatrix}
  =
  \begin{bmatrix}
    M+m & ml\cos(\theta)\\ ml\cos(\theta) & ml^2
  \end{bmatrix}^{-1}
  \left(
  \begin{bmatrix}
    ml\thetadot^2\sin(\theta)\\ mgl\sin(\theta)
  \end{bmatrix}
  +
  \begin{bmatrix}
    F \\ 0
  \end{bmatrix}
\right).
\end{equation}

So if we let out state be
\begin{equation}
  x = 
  \begin{bmatrix}
    z & \theta & \zdot & \thetadot
  \end{bmatrix}^T
\end{equation}
and we make the small angle approximation of 
\begin{equation}
  \begin{split}
    \cos(\theta) &\approx 1\\
    \sin(\theta) &\approx \theta \approx 0
  \end{split},
\end{equation}
we get some linearized equations
\begin{equation}
  \begin{split}
    (M+m)\zddot + ml\thetaddot &= F\\
    ml\zdot + ml^2\thetaddot - mgl\theta &= 0.
  \end{split}
\end{equation}
Solving for our state space equation we get
\begin{equation}
  \begin{split}
    \zddot &= \frac{F}{M} - \frac{mg}{M}\theta\\
    \thetaddot &= \frac{-F}{ml} + \frac{M+m}{ml}g\theta.
  \end{split}
\end{equation}
Forming this into the standard form, we get
\begin{equation}
\begin{split}
  \xdot &=
  \begin{bmatrix}
    0 & 0 & 1 & 0\\
    0 & 0 & 0 & 1\\
    0 & \frac{-mg}{m} & 0 & 0\\
    0 & \frac{M+m}{ml} & 0 & 0\\
  \end{bmatrix}x + 
  \begin{bmatrix}
    0 \\ 0 \\ \frac{1}{M} \\ \frac{-1}{ml}
  \end{bmatrix}F\\
  C &= % Usually this is denoted y...
  \begin{bmatrix}
    1 &0 &0& 0\\
    0 &1& 0& 0
  \end{bmatrix}
  x.
\end{split}
\end{equation}
\subsection{\matlab Simulation}

Insert diagram***

And maybe add a bit of context and description of what was done... 




% \bibliographystyle{ieeetr}
% \newcommand{\bibdir}{~}
% \bibliography{\bibdir/~}
\end{document}
