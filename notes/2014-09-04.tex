\documentclass[10pt]{article}
%\input{page}
\usepackage{fullpage}	% Does about the same as the \input{page}...
% ^ part of the preprint package...

\input{symbolmac}
\usepackage{amsmath}
\usepackage{amssymb}  	% provides Rbb, etc.
\usepackage{latexsym}  	% provides \Box
%\usepackage{theorem} \input{proof}
%\pagestyle{empty}
\usepackage{graphicx} 	% for inserting graphics.
%\usepackage{booktabs}	% for fancy tabular environments and stuff
%\usepackage{subfig}	% for subfigures... 
%(This breaks Beamer if the patch isn't applied!!)
%usepackage{hyperref}


%% ******** For code: ******************
%\usepackage{listings}
%\usepackage[usenames,dvipsnames]{color}
%% ******** Define the language: *******
%\lstdefinelanguage{matlabfloz}{%
%	alsoletter={...},%
%	morekeywords={% 						% keywords
%	break,case,catch,continue,elseif,else,end,for,function,global,%
%	if,otherwise,persistent,return,switch,try,while,...,
%        classdef,properties,methods},%
%	comment=[l]\%,% 						% comments
%	morecomment=[l]...,% 					% comments
%	morestring=[m]',%   						% strings 
%}[keywords,comments,strings]%
%\lstdefinestyle{matlab}{language=matlabfloz,							
%		keywordstyle=\color[rgb]{0,0,1},					
%		commentstyle=\color[rgb]{0.133,0.545,0.133},	% comments
%		stringstyle=\color[rgb]{0.627,0.126,0.941},	% strings
%		numbersep=3mm, numbers=left, numberstyle=\tiny% number style
%}
%% ******** Set the style: **************
%\lstset{style=matlab}

%\newcommand{\matlab}{{\sc Matlab }} % Or 
\renewcommand{\matlab}{{\sc Matlab }} 

% \newcommand{\homedir}{/home/aepound}
% \newcommand{\figdir}{\homedir/figs}  % picture stuff: ps,fig,tex
%\newtheorem{theorem}{Theorem}
%\newtheorem{corollary}{Corollary}
%\newtheorem{lemma}{Lemma}

\title{}
\author{}
\date{Sept. 2, 2014}



\begin{document}
\maketitle

%\begin{abstract}
%\end{abstract}
The things that we will be going over today:
\begin{enumerate}
\item State space of nonlinear systems
\item \emph{Local} linearization around an equilibrium point
\item \emph{local} linearization around a trajectory
\item Feedback linearization
\end{enumerate}

Side note: From last time, the small angle approximation \emph{should}
be:
\begin{equation}
  \begin{split}
    \thetadot &\approx 0\\
    \sin(\theta) & \approx \theta\\
    \cos(\theta) & \approx 1
  \end{split}
\end{equation}

\section{State Space of Nonlinear Systems}

State is denoted as $x \in \Rbb^n$.
Input is denoted as $u \in \Rbb^k$.
Output is denoted as $y \in \Rbb^m$.

The state space equations are 
\begin{align}
  \xdot(t) &= f(x,u)\\
  y(t)     &= g(x,u)
\end{align}

\subsection{Example: Inverted Pendulum}

The equations of motion of an inverted pendulum are:
\begin{equation}
  ml^2\thetaddot = mgl\sin(\theta) - b\thetadot + T,
\end{equation}
where $b$ is a coefficient of friction and $T$ is the input torque.
Letting $x =
\begin{bmatrix}
  \theta & \thetadot
\end{bmatrix}^T$, then we get the equation
\begin{equation}
  \xdot =
  \begin{bmatrix}
    \thetadot\\ \thetaddot
  \end{bmatrix} = f(x,u) = 
  \begin{bmatrix}
    \thetadot\\ \frac{g}{l}\sin(\theta) - \frac{b}{ml^2}\thetadot +
    \frac{u}{ml^2} 
  \end{bmatrix}
\end{equation}


\section{\emph{Local} Linearization around an equilibrium point}
\begin{description}
\item[Equilibrium point] Any point at which the states of the system
  are not changing.  Example: $\xdot = f(x_{eq}, u_{eq}) = 0$, and
  $y_{eq} = g(x_{eq},u_{eq})$.
\end{description}
Of note is that equilibrium points can be forced.
Now because the system is in a state of equilibrium around an
equilibrium point, there isn't much to do with the examining of thet
states, but we can look at the deviations from that state.

Thus, we will look at distrubing it from the equilibrium point.  Let 
\begin{equation}
  \begin{split}
    u(t) &= u_{eq} + \delta u(t) \quad \forall t \ge 0\\
    x(0) &= x_{eq} + \delta x_{eq}(t)\\
    \delta x(t) &\triangleq x(t) - x_{eq} \rightarrow x(t) = x_{eq} +
    \delta x(t). 
  \end{split}
\end{equation}



\end{document}