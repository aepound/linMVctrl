\documentclass[10pt]{article}
%\input{page}
\usepackage{fullpage}	% Does about the same as the \input{page}...
% ^ part of the preprint package...

\input{symbolmac}
\usepackage{amsmath}
\usepackage{amssymb}  	% provides Rbb, etc.
\usepackage{latexsym}  	% provides \Box
%\usepackage{theorem} \input{proof}
%\pagestyle{empty}
\usepackage{graphicx} 	% for inserting graphics.
%\usepackage{booktabs}	% for fancy tabular environments and stuff
%\usepackage{subfig}	% for subfigures... 
%(This breaks Beamer if the patch isn't applied!!)
%usepackage{hyperref}
\usepackage{enumitem}

%% ******** For code: ******************
%\usepackage{listings}
%\usepackage[usenames,dvipsnames]{color}
%% ******** Define the language: *******
%\lstdefinelanguage{matlabfloz}{%
%	alsoletter={...},%
%	morekeywords={% 						% keywords
%	break,case,catch,continue,elseif,else,end,for,function,global,%
%	if,otherwise,persistent,return,switch,try,while,...,
%        classdef,properties,methods},%
%	comment=[l]\%,% 						% comments
%	morecomment=[l]...,% 					% comments
%	morestring=[m]',%   						% strings 
%}[keywords,comments,strings]%
%\lstdefinestyle{matlab}{language=matlabfloz,							
%		keywordstyle=\color[rgb]{0,0,1},					
%		commentstyle=\color[rgb]{0.133,0.545,0.133},	% comments
%		stringstyle=\color[rgb]{0.627,0.126,0.941},	% strings
%		numbersep=3mm, numbers=left, numberstyle=\tiny% number style
%}
%% ******** Set the style: **************
%\lstset{style=matlab}

%\newcommand{\matlab}{{\sc Matlab }} % Or 
\renewcommand{\matlab}{{\sc Matlab }} 

% \newcommand{\homedir}{/home/aepound}
% \newcommand{\figdir}{\homedir/figs}  % picture stuff: ps,fig,tex
%\newtheorem{theorem}{Theorem}
%\newtheorem{corollary}{Corollary}
%\newtheorem{lemma}{Lemma}

\title{ECE 6320 Homework 4}
\author{Andrew Pound}
\date{\today}



\begin{document}
\maketitle

%\begin{abstract}
%\end{abstract}


\section{}
Find the fundamental and state transition matrix for
\begin{equation}
  \xdot(t) =
  \begin{bmatrix}
   0 & 1 \\ 0 & t
  \end{bmatrix}x(t)
\end{equation}
and 
\begin{equation}
  \xdot(t) =
  \begin{bmatrix}
    -1 & e^{2t} \\ 0 & -1 
  \end{bmatrix}x(t).
\end{equation}


\section{}

An oscillation can be generated by 
\begin{equation}
  \xdot(t) =
  \begin{bmatrix}
    0 & 1 \\ -1 & 0
  \end{bmatrix}x(t)
\end{equation}
Show that its solution is 
\begin{equation}
  x(t) =
  \begin{bmatrix}
    \cos(t)& \sin(t) \\ -\sin(t) & \cos(t)
  \end{bmatrix}x(0).
\end{equation}


\section{}
A communication satellite of mass $m$ orbiting around the earth is shown
in Figure 3. The altitude of the satellite is specified by $r(t)$,
$\theta(t)$, and $\phi(t)$ as shown. The orbit can be controlled by
three orthogonal thrusts; $u_r$, $u_\theta(t)$, and $u_\phi(t)$. If
you remember you have already computed the exact nonlinear equations
in homework 1. You can refresh your memory by looking equation (2.47)
on page 36 in Chen. One solution which corresponds to a circular orbit
is given by 
\begin{equation}
  \begin{split}
    x_0 &=
    \begin{bmatrix}
      r_0 & 0 & \omega_0t & \omega_0 & 0 & 0
    \end{bmatrix}^T\\
    u_0 &=
    \begin{bmatrix}
      0 & 0 & 0
    \end{bmatrix}^T.
  \end{split}
\end{equation}
It's linearized pertubation model at $x_0(t)$ and $u_0(t)$ (given
above) is given as
\begin{equation}
\begin{split}
  \dot{\xbar} &= A\xbar(t)\\
  &=
  \begin{bmatrix}
   0&1&0&0&0&0\\ 
   3\omega_0^2 &0&0&2\omega_0r_0 &0&0\\ 
   0&0&0&1&0&0\\ 
   0&-\frac{2\omega_0}{r_0} &0&0&0&0\\ 
   0&0&0&0&0&1\\ 
   0&0&0&0&-\omega_0^2 &0\\ 
  \end{bmatrix}\xbar(t).
\end{split}
\end{equation}

\begin{enumerate}[label=(\alph*)]
\item Compute the transition matrix ($e^{At}$) for the linearized
  system given above.
\item Use the parameters given in Homework 3 (param.m) to compute
  Jordan's form of $A$.
\item Based on the Jordan's form computed above what can you say about
  $e^{At}$ as $t \rightarrow \infinity$.
\item You can use \matlab (show your work step by step) for this work
  but first see if you can solve it yourself.
\end{enumerate}

\section{}
Compute $A^t$ and $e^{At}$ for the following matrices
\begin{align}
  A_1 &=
  \begin{bmatrix}
    1 & 1 & 0 \\ 0 & 1 & 0 \\ 0 & 0 & 1
  \end{bmatrix}
  & A_2 &=
  \begin{bmatrix}
    1 & 1 & 0 \\ 0 & 0 & 1 \\ 0 & 0 & 1
  \end{bmatrix}
\end{align}

% \bibliographystyle{ieeetr}
% \newcommand{\bibdir}{~}
% \bibliography{\bibdir/~}
\end{document}
