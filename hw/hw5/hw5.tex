\documentclass[10pt]{article}
%\input{page}
\usepackage{fullpage}	% Does about the same as the \input{page}...
% ^ part of the preprint package...

\input{symbolmac}
\usepackage{amsmath}
\usepackage{amssymb}  	% provides Rbb, etc.
\usepackage{latexsym}  	% provides \Box
%\usepackage{theorem} \input{proof}
%\pagestyle{empty}
\usepackage{graphicx} 	% for inserting graphics.
%\usepackage{booktabs}	% for fancy tabular environments and stuff
%\usepackage{subfig}	% for subfigures... 
%(This breaks Beamer if the patch isn't applied!!)
%usepackage{hyperref}
\usepackage{enumitem}

%% ******** For code: ******************
%\usepackage{listings}
%\usepackage[usenames,dvipsnames]{color}
%% ******** Define the language: *******
%\lstdefinelanguage{matlabfloz}{%
%	alsoletter={...},%
%	morekeywords={% 						% keywords
%	break,case,catch,continue,elseif,else,end,for,function,global,%
%	if,otherwise,persistent,return,switch,try,while,...,
%        classdef,properties,methods},%
%	comment=[l]\%,% 						% comments
%	morecomment=[l]...,% 					% comments
%	morestring=[m]',%   						% strings 
%}[keywords,comments,strings]%
%\lstdefinestyle{matlab}{language=matlabfloz,							
%		keywordstyle=\color[rgb]{0,0,1},					
%		commentstyle=\color[rgb]{0.133,0.545,0.133},	% comments
%		stringstyle=\color[rgb]{0.627,0.126,0.941},	% strings
%		numbersep=3mm, numbers=left, numberstyle=\tiny% number style
%}
%% ******** Set the style: **************
%\lstset{style=matlab}

%\newcommand{\matlab}{{\sc Matlab }} % Or 
\renewcommand{\matlab}{{\sc Matlab }} 

% \newcommand{\homedir}{/home/aepound}
% \newcommand{\figdir}{\homedir/figs}  % picture stuff: ps,fig,tex
%\newtheorem{theorem}{Theorem}
%\newtheorem{corollary}{Corollary}
%\newtheorem{lemma}{Lemma}

\title{ECE 6320 - Homework 5
}
\author{Andrew Pound}
\date{\today}



\begin{document}
\maketitle

%\begin{abstract}
%\end{abstract}



\section{}
Consider
\begin{equation*}
  \begin{split}
    \xdot(t) &=
    \begin{bmatrix}
      -1 & 5 \\ 0 & 2
    \end{bmatrix}x(t) +
    \begin{bmatrix}
      2 \\ 0
    \end{bmatrix}u(t)\\
    y(t) &= 
    \begin{bmatrix}
      -2 & 4
    \end{bmatrix}x(t) - 2u(t).
  \end{split}
\end{equation*}
Is it BIBO stable?

Well, Let's take a look at the transfer function for this system.
\begin{equation*}
\begin{split}
  \hat{G}(s)
  &= C (sI - A)^{-1}B + D\\
  &=
  \begin{bmatrix}
    -2 & 4
  \end{bmatrix}
  \begin{bmatrix}
    s+1 & 5 \\ 0 & s -2
  \end{bmatrix}^{-1}
  \begin{bmatrix}
    2 \\ 0
  \end{bmatrix} - 2 \\
  &= \frac{1}{(s+1)(s-2)}
  \begin{bmatrix}
    -2 & 4
  \end{bmatrix}
  \begin{bmatrix}
    s-2 & -5 \\ 0 & s+1
  \end{bmatrix}
  \begin{bmatrix}
    2 \\ 0
  \end{bmatrix} - 2\\
  &= \frac{1}{(s+1)(s-2)}
  \begin{bmatrix}
    -2 & 4
  \end{bmatrix}
  \begin{bmatrix}
    2(s-2) \\ 0
  \end{bmatrix} - 2\\
 &= \frac{1}{(s+1)(s-2)} \left(-4\left(s-2\right)\right) - 2\\
 &=\frac{-4(s-2)}{(s+1)(s-2)} - 2\\
 &=\frac{-4}{s+1} + \frac{-2s -2}{s+1}\\
 &=\frac{-2s -6}{s+1}.
\end{split}
\end{equation*}
Now we can see from looking at $\hat{G}(s)$, that the transfer
function is proper, and than the pole is located in the Left Half
Plane (LHP).  That is, the pole is $s = -1$.  Because of this, we can
say that the system is BIBO stable.

\section{}
Consider a system 
\begin{equation*}
  \begin{split}
    \xdot(t) &=
    \begin{bmatrix}
      -2 & 0 & 0 \\ 0 & 1 & 0 \\ 0 & 0 & -1
    \end{bmatrix}x(t) +
    \begin{bmatrix}
      1 \\ 0 \\ -1
    \end{bmatrix}u(t)\\
    y(t) &= 
    \begin{bmatrix}
      1 & 1 & 0
    \end{bmatrix}x(t) + u(t).
  \end{split}
\end{equation*}
\begin{enumerate}[label=(\alph*)]
\item Compute the system transfer function.
\item[]
\begin{equation*}
\begin{split}
\hat{G}(s) &=
\begin{bmatrix}
  1 & 1 & 0
\end{bmatrix}
\begin{bmatrix}
  s+2 & 0 & 0 \\ 0 & s-1 & 0 \\ 0 & 0 & s+1
\end{bmatrix}^{-1}
\begin{bmatrix}
  1 \\ 0 \\ -1
\end{bmatrix} + 1\\
&= 
\begin{bmatrix}
  1 & 1 & 0
\end{bmatrix}
\begin{bmatrix}
  \frac{1}{s+2} & 0 & 0 \\ 0 & \frac{1}{s-1} & 0 \\ 0 & 0 & \frac{1}{s+1}
\end{bmatrix}^{-1}
\begin{bmatrix}
  1 \\ 0 \\ -1
\end{bmatrix} + 1\\
& =
\begin{bmatrix}
  1 & 1 & 0
\end{bmatrix}
\begin{bmatrix}
  \frac{1}{s+2} \\ 0 \\ \frac{1}{s+1}
\end{bmatrix}+1\\
& = \frac{1}{s+2} + 1 \\ 
&= \frac{1}{s+2} + \frac{s+2}{s+2} \\
&= \frac{s+3}{s+2}.
\end{split}
\end{equation*}
\item Is the matrix $A$ asymptotically stable, marginally stable, or unstable?
This is marginally unstable, because the matrix $A$ has an eigenvalue
of $\lambda = +1$.
\item Is the system BIBO stable?
This transfer function is proper and the pole is at $s= -2$, so it is
BIBO stable.

\end{enumerate}


\section{}
Is the homogeneous state equation 
\begin{equation*}
  \xdot (t)  =
  \begin{bmatrix}
    -1 & 0 & 2\\ 0 & 0 & 1 \\ 0 & 0 & 0
  \end{bmatrix}x(t)
\end{equation*}
marginally stable? Asymptotically stable?

Because the matrix $A$ is in triangular form, we can read off the
eigenvalues as $\lambda_i = -1, 0, 0$.  Thus the eigenvalue(s) of $0$
are not simple.  Thus, we can determine that the system is both
marginally and asymptotically unstable.

\section{}
Is the discrete-time homogenous stat equation
\begin{equation*}
  x(k+1) =
  \begin{bmatrix}
    0.9 & 0 & 2\\ 0 & 1 & 0\\ 0 & 0 & 1
  \end{bmatrix}x(k)
\end{equation*}
marginally stable? Asymptotically stable?

The Jordan form for this matrix is
\begin{equation*}
  J =
  \begin{bmatrix}
    0.9 & 0 & 0 \\ 0 & 1 & 0\\ 0 & 0 & 1
  \end{bmatrix}.
\end{equation*}
The eigenvalues are then $\lambda_i = 0.9, 1, 1$, and the $1$'s are
distinct, because there are no super diagonal ones.  Thus, we can say
that the system is marginally stable, though not asymptotically
stable.





\section{}
Consider the linear time varying equation
\begin{equation}
  \begin{split}
    \xdot(t) &= 2tx(t) + u(t) \\
    y(t) &= e^{-t^2}x(t).
  \end{split}
\end{equation}
Is the equation BIBO stable? Marginally stable? Asymptotically stable?

\section{}
Conside a LTV system with a state transition matrix $\Phi(t,\tau)$ for
which 
\begin{equation*}
  \Phi(t,0) =
  \begin{bmatrix}
    e^t\cos(2t) & e^{-2t}\sin(2t) \\ 
    -e^{t}\sin(2t) & e^{-2t}\cos(2t)
  \end{bmatrix}.
\end{equation*}
\begin{enumerate}[label=(\alph*)]
\item Compute the state transition matrix $\Phi(t,t_0)$.
\item[]
  Ok, to do this we will use the following properties:
  \begin{align}
    \phi(t_2, t_1)\phi(t_1,t_0) &= \phi(t_2, t_0)  & 
    \phi^{-1}(t,\tau) &= \phi(\tau, t).
  \end{align}
Consider that we know $\phi(t,0)$, then let's look at 
\begin{equation*}
  \phi(t,t_0)\phi(t_0,0) = \phi(t,0).
\end{equation*}
Solve for what we want,
\begin{equation*}
  \phi(t,t_0) = \phi(t,0)\phi(0,t_0)^{-1}
\end{equation*}
\begin{align*}
  \phi(t_0,0) & =
  \begin{bmatrix}
    e^{t_0}\cos(2t_0) & e^{-2t_0}\sin(2t_0) \\ 
    -e^{t_0}\sin(2t_0) & e^{-2t_0}\cos(2t_0)
  \end{bmatrix} \\
  \phi(t_0,0)^{-1} & =\frac{1}{\det\left(\phi(t_0,0)\right)}
  \begin{bmatrix}
    e^{-2t_0}\cos(2t_0) & -e^{-2t_0}\sin(2t_0) \\ 
    e^{t_0}\sin(2t_0) & e^{t_0}\cos(2t_0)
  \end{bmatrix} \\
  &=\frac{1}{e^{-t_0}}
  \begin{bmatrix}
    e^{-2t_0}c_0 & -e^{-2t_0}s_0 \\
    e^{t_0}s_0 & e^{t_0}c_0
  \end{bmatrix}\\
  &=
  \begin{bmatrix}
    e^{-t_0}c_0 & -e^{-t_0}s_0 \\
    e^{2t_0}s_0 & e^{2t_0}c_0
  \end{bmatrix},
\end{align*}
where $c_0 = \cos(2t_0)$, and $s_0 = \sin(2t_0)$. Now we can plug back
in to get the full transition matrix
\begin{equation*}
  \begin{split}
    \phi(t,t_0) &= \phi(t,0)\phi^{-1}(t_0, 0)\\
    &=
    \begin{bmatrix}
      e^tc & e^{-2t} s \\ -e^ts & e^{-2t}c
    \end{bmatrix}
    \begin{bmatrix}
      e^{-t_0}c_0 & -e^{-t_0}s_0 \\
      e^{2t_0}s_0 & e^{2t_0}c_0
    \end{bmatrix}\\
    &= 
    \begin{bmatrix}
      e^{t-t_0}cc_0 + e^{-2(t-t_0)}ss_0 & -e^{t-t_0}cs_0 + e
      ^{-2(t-t_0)}s c_0\\
      -e^{t-t_0}sc_0 +e^{-2(t-t_0)}cs_0 & e^{t-t_0}ss_0 +
      e^{-2(t-t_0)}cc_0 
    \end{bmatrix}
  \end{split}
\end{equation*}
\item Compute a matrix $A(t)$ that corresponds to the given state
  transition matrix.
\item Compute the eigenvalues of $A(t)$.
\item Classify this system in terms of Lyapunov stability (marginally
  stable, asymptotically stable, or exponentially stable).
\end{enumerate}

Hint: In answering part (d), do not be misled by your answer of part
(c)


\section{}
Investigate whether or not the solutions to the following nonlinear
systems converge to the equilibrium point when they start close enough
to it.

\begin{enumerate}[label=(\alph*)]
\item The state space system 
  \begin{equation*}
    \begin{split}
      \xdot_1 &= -x_1 + x_1\left(x_1^2 + x_2^2\right)\\
      \xdot_2 &= -x_2 + x_2\left(x_1^2 + x_2^2\right)
    \end{split}
  \end{equation*}
with equilibrium points $x_1 = x_2 = 0$.
\item The second order system $\ddot{w} + g(w)\wdot + w = 0$ with
  equilibrium point $w = \wdot = 0$. Remember this is not the state
  space representation (you can put it into sate space form by
  selecting $x_1 = w$ and $x_2 = \wdot$). Determine for which values
  of $g(0)$ we can gaurantee convergence to the origin based on the
  local linearization.
\end{enumerate}

\section{}
Consider the continuous-time LTI system
\begin{equation*}
  \xdot = Ax, \quad x \in \Rbb^n
\end{equation*}
and suppose that there exists a positive constant $\mu$ and
positive-definite matrices $P,Q \; \in \Rbb^n$ for the Lyapunov
equation
\begin{equation*}
  A^TP + PA + 2\mu P = -Q.
\end{equation*}
Show that all the eigenvalues of $A$ have reall parts less than
$-\mu$. A matrix $A$ with this property is said to be
\emph{asymptotically stable with stability margin $\mu$}.

\emph{Hint: Start by showing that all the eigenvalues of $A$ have real
  parts less than $-\mu$ if and only if all eigenvalues of $A+\mu I$
  have real parts less than $0$. (i.e. $A+\mu I$ is a stability
  matrix). 
% \bibliographystyle{ieeetr}
% \newcommand{\bibdir}{~}
% \bibliography{\bibdir/~}
\end{document}
