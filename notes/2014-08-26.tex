\documentclass[10pt]{article}
%\input{page}
\usepackage{fullpage}	% Does about the same as the \input{page}...
% ^ part of the preprint package...

\input{symbolmac}
\usepackage{amsmath}
\usepackage{amssymb}  	% provides Rbb, etc.
\usepackage{latexsym}  	% provides \Box
%\usepackage{theorem} \input{proof}
%\pagestyle{empty}
\usepackage{graphicx} 	% for inserting graphics.
%\usepackage{booktabs}	% for fancy tabular environments and stuff
%\usepackage{subfig}	% for subfigures... 
%(This breaks Beamer if the patch isn't applied!!)
%usepackage{hyperref}


%% ******** For code: ******************
%\usepackage{listings}
%\usepackage[usenames,dvipsnames]{color}
%% ******** Define the language: *******
%\lstdefinelanguage{matlabfloz}{%
%	alsoletter={...},%
%	morekeywords={% 						% keywords
%	break,case,catch,continue,elseif,else,end,for,function,global,%
%	if,otherwise,persistent,return,switch,try,while,...,
%        classdef,properties,methods},%
%	comment=[l]\%,% 						% comments
%	morecomment=[l]...,% 					% comments
%	morestring=[m]',%   						% strings 
%}[keywords,comments,strings]%
%\lstdefinestyle{matlab}{language=matlabfloz,							
%		keywordstyle=\color[rgb]{0,0,1},					
%		commentstyle=\color[rgb]{0.133,0.545,0.133},	% comments
%		stringstyle=\color[rgb]{0.627,0.126,0.941},	% strings
%		numbersep=3mm, numbers=left, numberstyle=\tiny% number style
%}
%% ******** Set the style: **************
%\lstset{style=matlab}

%\newcommand{\matlab}{{\sc Matlab }} % Or 
\renewcommand{\matlab}{{\sc Matlab }} 

% \newcommand{\homedir}{/home/aepound}
% \newcommand{\figdir}{\homedir/figs}  % picture stuff: ps,fig,tex
%\newtheorem{theorem}{Theorem}
%\newtheorem{corollary}{Corollary}
%\newtheorem{lemma}{Lemma}

\title{}
\author{}
\date{Aug. 26, 2014}



\begin{document}
\maketitle

%\begin{abstract}
%\end{abstract}

\section{State Space Representation}
State is denoted as $x \in \Rbb^n$.
Input is denoted as $u \in \Rbb^k$.
Output is denoted as $y \in \Rbb^m$.

The state space equations are 
\begin{align}
  \xdot(t) &= A(t) x(t) + B(t)u(t)\\
  y(t)     &= C(t) x(t) + D(t)u(t)
\end{align}
where 
\begin{equation}
  \begin{split}
    A &\in \Rbb^{n\times n}\\
    B &\in \Rbb^{n\times k}\\
    C &\in \Rbb^{m\times n}\\
    D &\in \Rbb^{m\times k}.\\
  \end{split}
\end{equation}
All the matrices $ A,B,C,D$ are matrices and (possibly) time-varying.
If they are constant with respect to time, $t$, then this is a
continuos time LTI system.

Also have a discrete time version:
\begin{align}
  \xdot(t+1) &= A(t) x(t) + B(t)u(t)\\
  y(t)     &= C(t) x(t) + D(t)u(t)
\end{align}
for $t$ at discrete time instances.

An example:

% Insert figure here...
% Free body diagramming

\begin{equation}
%  \left.
  \begin{split}
    m_1\yddot &= u_1 - k_1y_1 - k_2(y_1 - y_2)\\
    m_1\yddot &= u_2 - k_1y_2 + k_2(y_1 - y_2).
  \end{split} % \right}\text{equations of motion.}
\end{equation}
These are the equations of motion.

We can define our states
\begin{equation}
  \begin{split}
    x &=
    \begin{bmatrix}
      y_1 & y_2 & \ydot_1 & \ydot_2
    \end{bmatrix}^T\\
    u &=
    \begin{bmatrix}
      u_1 & u_2
    \end{bmatrix}^T    
  \end{split}
\end{equation}
We want to represent this in the form:
\begin{equation}
  \xdot = Ax + Bu.
\end{equation}

\begin{equation}
  \begin{split}
    \yddot_1 &= \frac{u_1}{m_1} - \frac{(k_1 + k_2)}{m_1}y_1 +
    \frac{k_2}{m_1}y_2\\ 
    \yddot_2 &= \frac{u_2}{m_2} + \frac{k_2}{m_2} y_1 - \frac{(k_1 +
      k_2)}{m_2} y_2.
  \end{split}
\end{equation}
Then the equation of motion becomes
\begin{equation}
\begin{split}
  \xdot =
  \begin{bmatrix}
    \ydot_1 \\ \ydot_2 \\ \yddot_1 \\ \yddot_2
  \end{bmatrix}
  &=
  \begin{bmatrix}
    0 & 0 & 1 & 0\\ 0 & 0 & 0 & 1 \\
    \frac{-(k_1 + k_2)}{m_1} & \frac{k_2}{m_1} & 0 & 0\\
    \frac{k_2}{m_2} &  \frac{-(k_1 + k_2)}{m_2} & 0 & 0
  \end{bmatrix}
  \begin{bmatrix}
    y_1 \\ y_2 \\ \ydot_1 \\ \ydot_2
  \end{bmatrix}
  +
  \begin{bmatrix}
    0 & 0 \\ 0 & 0 \\ \frac{1}{m_1} & 0 \\ 0 & \frac{1}{m_2}
  \end{bmatrix}
  \begin{bmatrix}
    u_1 \\ u_2
  \end{bmatrix}\\
y =
\begin{bmatrix}
  y_1 \\ y_2
\end{bmatrix}
& = 
\begin{bmatrix}
  1& 0 & 0 & 0 \\ 0 & 1 & 0 & 0
\end{bmatrix}
  \begin{bmatrix}
    y_1 \\ y_2 \\ \ydot_1 \\ \ydot_2
  \end{bmatrix}
+ \begin{bmatrix}
    0 & 0 \\ 0 & 0 \\0 & 0 \\ 0 & 0
  \end{bmatrix}
  \begin{bmatrix}
    u_1 \\ u_2
  \end{bmatrix}.
\end{split}
\end{equation}

The states are \emph{not} unique, but the order of the states
(i.e. the number of the states) is unique.

\section{Lagrange's Equations}
\begin{itemize}
\item Useful for deriving complex mechanical systems equations
\item Differential Equations
\item Automatically incorporates the constraints imposed by the
  different interacting system
\item disadvantages: Sometimes it can be difficult to choose your
  state space correctly or carefully and a bad choice will mess things
  up. 
\end{itemize}

\subsection{Generalized Coordinates}
We define the generalized coordinates as 
$q =
\begin{bmatrix}
  q_1 & \dots & r_r
\end{bmatrix}^T$ where there are 
 one for each independent degree of freedom.

Kinetic energy, $T$, as a function of the $q_i$'s and the
$\qdot_i$'s. And The potential energy, $V$, is a funciton of $q_i$'s.

The Lagrangian function is given as
\begin{equation}
  L = T - V.
\end{equation}
The the desired equations of motion can be derived as
\begin{equation}
  \frac{d}{dt} \left(\partiald{L}{q_i}\right) - \partiald{L}{q_i} = \Theta_i,
\end{equation}
where $\Theta_i$ is a force.

\subsection{Example}

% Insert figure here...

In this example, there are only 2 degrees of motion:
\begin{equation}
  q =
  \begin{bmatrix}
    z_1 & z_2
  \end{bmatrix}^T.
\end{equation}
Then we can calculate the Kinetic Energy as
\begin{equation}
  T = \frac1 2 m_1\zdot^2_1 + \frac{1}{2}m_2 \zdot^2_2,
\end{equation}
and the Potential Energy as 
\begin{equation}
  V = \frac{1}{2}k\left(z_1 - z_2\right)^2.
\end{equation}
Then, we calculate the Lagrangian as
\begin{equation}
  L = T - V,
\end{equation}
And solve for our equations of motion:
\begin{equation}
  \frac{d}{dt}\left(m_1 \zdot_1\right) - k\left(z_1 - z_2\right) = F,
\end{equation}
Which gives
\begin{equation}
  \begin{split}
    m_1\zddot_1 -k\left(z_1 - z_2\right) &= F_1\\
    m_2\zddot_2 + k\left(z_1 - z_2\right) &= F_2.
  \end{split}
\end{equation}

Next class: Pendulum on a cart...

\end{document}