\documentclass[10pt]{article}
%\input{page}
\usepackage{fullpage}	% Does about the same as the \input{page}...
% ^ part of the preprint package...

\input{symbolmac}

\usepackage{amsmath}
\usepackage{amssymb}  	% provides Rbb, etc.
\usepackage{latexsym}  	% provides \Box
%\usepackage{theorem} \input{proof}
%\pagestyle{empty}
\usepackage{graphicx} 	% for inserting graphics.
%\usepackage{booktabs}	% for fancy tabular environments and stuff
%\usepackage{subfig}	% for subfigures... 
%(This breaks Beamer if the patch isn't applied!!)
%usepackage{hyperref}


%% ******** For code: ******************
%\usepackage{listings}
%\usepackage[usenames,dvipsnames]{color}
%% ******** Define the language: *******
%\lstdefinelanguage{matlabfloz}{%
%	alsoletter={...},%
%	morekeywords={% 						% keywords
%	break,case,catch,continue,elseif,else,end,for,function,global,%
%	if,otherwise,persistent,return,switch,try,while,...,
%        classdef,properties,methods},%
%	comment=[l]\%,% 						% comments
%	morecomment=[l]...,% 					% comments
%	morestring=[m]',%   						% strings 
%}[keywords,comments,strings]%
%\lstdefinestyle{matlab}{language=matlabfloz,							
%		keywordstyle=\color[rgb]{0,0,1},					
%		commentstyle=\color[rgb]{0.133,0.545,0.133},	% comments
%		stringstyle=\color[rgb]{0.627,0.126,0.941},	% strings
%		numbersep=3mm, numbers=left, numberstyle=\tiny% number style
%}
%% ******** Set the style: **************
%\lstset{style=matlab}

%\newcommand{\matlab}{{\sc Matlab }} % Or 
\renewcommand{\matlab}{{\sc Matlab }} 

% \newcommand{\homedir}{/home/aepound}
% \Newcommand{\figdir}{\homedir/figs}  % picture stuff: ps,fig,tex
%\newtheorem{theorem}{Theorem}
%\newtheorem{corollary}{Corollary}
%\newtheorem{lemma}{Lemma}

\usepackage{wrapfig}

\title{}
\author{}
\date{Sept. 2, 2014}



\begin{document}
\maketitle

Note: There was a minus sign error in HW6.

\section{Next HW: HW7}
You will need to edit just 2 files: \emph{param.m} and
\emph{differential\_control file} (Don't know the name exactly, but
it's what does the controlling...).

Start with \emph{param.m} and fill in where there are question marks.

Also of note: there is a function to do the $z = R(\psi)u_p$ part (as
on slide 12 in lecture slides/notes) - it is implemented in
\emph{Rv\_v1.m}. 

\section{Review for test on Tuesday}

\subsection{Prob 2 on practice exam}
\begin{equation}
  \xdot = A(t)x, \qquad x(0) = x_0
\end{equation}
We also know the state transition matrix $\phi(t,\tau)$.
Then this is fed in as the input to another system
\begin{equation}
  \zdot = A(t)z + x(t), \qquad z(0) = z_0.
\end{equation}
Compute x(t) and z(t) as a function of $x_0$ and $z_0$.

\paragraph{Answer}
This is an LTV system, so the solution of the first system is given by 
\begin{equation}
  x(t) = \phi(t,0)x_0.
\end{equation}
\begin{wrapfigure}[11]{r}{.4\textwidth}
\fbox{\begin{minipage}{\dimexpr\linewidth-2\fboxrule-2\fboxsep}
 Remember: The state transition matrix $\phi(t,\tau)$ is always found
 on the  homogeneous system  (i.e. with zero input).  Then the general
 solution is
 \begin{equation*}
   x(t) = \phi(t,0)x_0 + \int_0^t\phi(t,\tau)B(\tau)d\tau.
 \end{equation*}
\end{minipage}}
\end{wrapfigure}

Then we got
\begin{equation*}
  \zdot = A(t)z + x(t).
\end{equation*}
And because this uses the same $A$ matrix, we know that the solution
is just the same as for the $x(t)$ system
\begin{equation*}
  z(t) = \phi(t,0)z_0 + \int_0^t\phi(t,\tau)x(\tau)d\tau.
\end{equation*}
And plugging in for $x(t)$, we get
\begin{equation*}
  \begin{split}
    z(t) &= \phi(t,0)z_0 + \int_0^t\phi(t,\tau)\phi(\tau,0)x_0d\tau \\
    &= \phi(t,0)z_0 + \int_0^t\phi(t,\tau)\phi(\tau,0)x_0d\tau\\
    &= \phi(t,0)z_0 + \int_0^t\phi(t,0)x_0d\tau\\
    &= \phi(t,0)z_0 + \phi(t,0)x_0\int_0^td\tau\\
    &= \phi(t,0)z_0 + \phi(t,0)x_0t.
  \end{split}
\end{equation*}

\subsection{Linearity}

If $y = f(x)$, and $f$ is linear with respect to $x$, then we can say
that 
\begin{align*}
  x &= c_1x_1 + c_2x_2 \qquad \qquad\qquad \Rightarrow  & y &= c_1f(x_1) + c_2f(x_2).
\end{align*}

\subsection{LTI systems}
\begin{equation*}
  \begin{split}
    \xdot &= Ax(t) + Bu(t) \\
    y  &= Cx(t) + Du(t)
  \end{split}
\end{equation*}
First, find the solution to the Homogeneous problem $\xdot = Ax(t)$:
\begin{equation*}
  x(t) = e^{At}x_0.
\end{equation*}
Side note: to then find $A$, differentiate:
\begin{equation*}
  A = \left.\frac{d}{dt} e^{At}\right|_{t=0} = \left.Ae^{At}\right|_{t=0}.
\end{equation*}

\paragraph{solution}
\begin{equation*}
  x(t) = e^{Ar}x_0 + \int_0^te^{A(t-\tau)}Bu(\tau)d\tau
\end{equation*}
The transfer function is:
\begin{equation*}
  \hat{g}(s) = C(sI - A)^{-1}B + D.
\end{equation*}
The inpulse response:
\begin{equation*}
  g(t) = \mathcal{L}^{-1}(\hat(g)(s)).
\end{equation*}
May want to look over the derivation of this...
\begin{equation*}
  y(t) = \int_0^tg(t-\tau)u(\tau)d\tau
\end{equation*}
and 
\begin{equation*}
  \hat{y}(s) = \hat{g}(s)\hat{u}(s).
\end{equation*}


\subsection{LTV systems}
\begin{equation*}
  \begin{split}
    \xdot &= A(t)x(t) + B(t)u(t) \\
    y  &= C(t)x(t) + D(t)u(t)
  \end{split}
\end{equation*}

There are no transfer functions. and the eigenvalues don't indicate
stability. 

\paragraph{solution}
\begin{equation*}
  y(t) = C(t) \phi(t,t_0)x_o +
  \int_{t_0}^t\phi(t,\tau)B(\tau)u(\tau)d\tau
\end{equation*}

\begin{equation*}
  \xdot(t) = A(t)x(t) 
\end{equation*}
has size $n$.

Choose Linearly independent initial conditions at $t =0$, of which we
need $n$, to be able to form the fundamental matrix $X(t)$.
Thus,
\begin{align*}
  x_{01} &\rightarrow x(t)_1\\
  x_{02} &\rightarrow x(t)_2
\end{align*}
and 
\begin{equation*}
  X(t) =
  \begin{bmatrix}
    x(t)_1 & x(t)_2
  \end{bmatrix}.
\end{equation*}
Then find $X^{-1}(t_0)$, to find
\begin{equation*}
  \phi(t,t_0) = X(t) X^{-1}(t_0).
\end{equation*}
Some properties of the state transistion matrix $\phi(t,t_0)$:
\begin{align*}
  \phi(t,t) &= I & \frac{d}{dt}\phi(t,t_0) = A\phi(t,t_0)\\
  \phi(\tau,\tau) &= I 
\end{align*}

\subsection{Linearization}
\begin{align*}
  \xdot &= f(x,u) \\
  y &= h(x)
\end{align*}
Then we want to look at the solution around some point
$(x^{eq},u^{eq})$, or trajectory $(x^{traj},u^{traj})$.
Then we can define the following variables and state equations:
\begin{align*}
  \delta x &= x - x^{eq}&   \delta\xdot &= A\delta x + B \delta u\\
  \delta u &= u - u^{eq}&   \delta y &= C\delta x
  \delta y &= y - y^{eq}&
\end{align*}
where
\begin{align*}
  A &= \left.\partiald{f}{x}\right|_{(x^{eq},u^{eq})}&
  B &= \left.\partiald{f}{u}\right|_{(x^{eq},u^{eq})}\\
  C &= \left.\partiald{h}{x}\right|_{(x^{eq},u^{eq})}
\end{align*}

\subsection{Stability}

\subsubsection{Internal/Lyapunov}
Three kinds:
\begin{enumerate}
\item Marginally stable
\item Exponentiall/Asymptotically stable
\item Unstable
\end{enumerate}

\paragraph{Marginal stability}
Look at the eigenvalues:
$\text{real}(\lambda_i) < 0$, and $\lambda_i = 0$ are all simple
roots. 
For Dicrete time systems, $\left|\lambda_i\right| < 1$. 

Lyapunov constraints:
\begin{equation*}
  PA + A^TP = -Q, \qquad \forall \text{Positive Definite } Q. 
\end{equation*}

( We may need to find Jordan form for $2\times 2$ or $3\times 3$
matrices iff $\lambda$'s $= 0$.)
\paragraph{Asymptotic stability}
$x\rightarrow 0$.


\paragraph{Instability}
Unbounded...

\subsubsection{BIBO stability}
\begin{equation*}
  y(t) = \int_0^\infinity g(\tau) u(t - \tau) d\tau 
\end{equation*}
if $\left\Vert u(t)\right\Vert \le M<\infinity$.
If something is internally stable, then it is BIBO stable.
\paragraph{Time domain criteria}
\begin{equation*}
  \int_0^\infinity \left|g(\tau)\right|d\tau < \infinity.
\end{equation*}

\paragraph{S-domain criteria}
\begin{equation*}
  g(s) = \frac{d(s)}{n(s)}
\end{equation*}
Consider the poles of $n(s)$, and if all the noncanceled poles of
$n(s)$ are strictly in the LHP, then it it BIBO stable.



\end{document}