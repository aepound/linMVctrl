\documentclass[10pt]{article}
%\input{page}
\usepackage{fullpage}	% Does about the same as the \input{page}...
% ^ part of the preprint package...

\input{symbolmac}
\usepackage{amsmath}
\usepackage{amssymb}  	% provides Rbb, etc.
\usepackage{latexsym}  	% provides \Box
%\usepackage{theorem} \input{proof}
%\pagestyle{empty}
\usepackage{graphicx} 	% for inserting graphics.
%\usepackage{booktabs}	% for fancy tabular environments and stuff
%\usepackage{subfig}	% for subfigures... 
%(This breaks Beamer if the patch isn't applied!!)
%usepackage{hyperref}


%% ******** For code: ******************
%\usepackage{listings}
%\usepackage[usenames,dvipsnames]{color}
%% ******** Define the language: *******
%\lstdefinelanguage{matlabfloz}{%
%	alsoletter={...},%
%	morekeywords={% 						% keywords
%	break,case,catch,continue,elseif,else,end,for,function,global,%
%	if,otherwise,persistent,return,switch,try,while,...,
%        classdef,properties,methods},%
%	comment=[l]\%,% 						% comments
%	morecomment=[l]...,% 					% comments
%	morestring=[m]',%   						% strings 
%}[keywords,comments,strings]%
%\lstdefinestyle{matlab}{language=matlabfloz,							
%		keywordstyle=\color[rgb]{0,0,1},					
%		commentstyle=\color[rgb]{0.133,0.545,0.133},	% comments
%		stringstyle=\color[rgb]{0.627,0.126,0.941},	% strings
%		numbersep=3mm, numbers=left, numberstyle=\tiny% number style
%}
%% ******** Set the style: **************
%\lstset{style=matlab}

%\newcommand{\matlab}{{\sc Matlab }} % Or 
\renewcommand{\matlab}{{\sc Matlab }} 

\providecommand{\norm}[1]{\lVert#1\rVert}

% \newcommand{\homedir}{/home/aepound}
% \newcommand{\figdir}{\homedir/figs}  % picture stuff: ps,fig,tex
%\newtheorem{theorem}{Theorem}
%\newtheorem{corollary}{Corollary}
%\newtheorem{lemma}{Lemma}

\title{}
\author{}
\date{Sept. 2, 2014}



\begin{document}
\maketitle

%\begin{abstract}
%\end{abstract}
The things that we will be going over today:
\begin{enumerate}
\item State space of nonlinear systems
\item \emph{Local} linearization around an equilibrium point
\item \emph{local} linearization around a trajectory
\item Feedback linearization
\end{enumerate}

Side note: From last time, the small angle approximation \emph{should}
be:
\begin{equation}
  \begin{split}
    \thetadot &\approx 0\\
    \sin(\theta) & \approx \theta\\
    \cos(\theta) & \approx 1
  \end{split}
\end{equation}

\section{State Space of Nonlinear Systems}

State is denoted as $x \in \Rbb^n$.
Input is denoted as $u \in \Rbb^k$.
Output is denoted as $y \in \Rbb^m$.

The state space equations are 
\begin{align}
  \xdot(t) &= f(x,u)\\
  y(t)     &= g(x,u)
\end{align}

\subsection{Example: Inverted Pendulum}

The equations of motion of an inverted pendulum are:
\begin{equation}
  ml^2\thetaddot = mgl\sin(\theta) - b\thetadot + T,
\end{equation}
where $b$ is a coefficient of friction and $T$ is the input torque.
Letting $x =
\begin{bmatrix}
  \theta & \thetadot
\end{bmatrix}^T$, then we get the equation
\begin{equation}
  \xdot =
  \begin{bmatrix}
    \thetadot\\ \thetaddot
  \end{bmatrix} = f(x,u) = 
  \begin{bmatrix}
    \thetadot\\ \frac{g}{l}\sin(\theta) - \frac{b}{ml^2}\thetadot +
    \frac{u}{ml^2} 
  \end{bmatrix}
\end{equation}


\section{\emph{Local} Linearization around an equilibrium point}
\begin{description}
\item[Equilibrium point] Any point at which the states of the system
  are not changing.  Example: $\xdot = f(x_{eq}, u_{eq}) = 0$, and
  $y_{eq} = g(x_{eq},u_{eq})$.
\end{description}
Of note is that equilibrium points can be forced.
Now because the system is in a state of equilibrium around an
equilibrium point, there isn't much to do with the examining of thet
states, but we can look at the deviations from that state.

Thus, we will look at distrubing it from the equilibrium point.  Let 
\begin{equation}
  \begin{split}
    u(t) &= u_{eq} + \delta u(t) \quad \forall t \ge 0\\
    x(0) &= x_{eq} + \delta x_{eq}(t)\\
    \delta x(t) &\triangleq x(t) - x_{eq} \rightarrow x(t) = x_{eq} +
    \delta x(t). 
  \end{split}
\end{equation}

\newcommand{\meq}[1]{#1^{eq}}

Ok, let $u(t) = \meq{u} + \delta u$.
\begin{equation}
  \begin{split}
    \delta y &= y(t) - \meq{y} \\ 
    &= g(x,u) - g(\meq{x},\meq{u})\\
    &= g(\meq{x}+\delta x,\meq{u}+ \delta u) - g(\meq{x},\meq{u}).
  \end{split}
\end{equation}
We can do a series expansion to get
\begin{equation}\label{eq:delty}
\delta y =
\underbrace{\partiald{g(\meq{x},\meq{u})}{x}}_{\text{Jacobian $(m
    \times n)$}}\delta x +
\partiald{g(\meq{x},\meq{u})}{u}\delta u + O(\norm{\delta x}^2) +
O(\norm{\delta u}^2),
\end{equation}

Note: This is analysis is not a state space analysis now.  It is a
deviation dynamics problem.  We are now looking at deviations and
perturbations around the equilibrium point, \emph{NOT} states,
anymore. 

The Jacobian is defined as 
\begin{equation}
  \partiald{g}{x} = \left[ \left(\partiald{g_i}{x_j}\right)_{ij}\right].
\end{equation}

Just like equation \ref{eq:delty}, we can do the same analysis on the
state equation
\begin{equation}
\delta y =
\underbrace{\partiald{f(\meq{x},\meq{u})}{x}}_{\text{Jacobian $(m
    \times n)$}}\delta x +
\partiald{f(\meq{x},\meq{u})}{u}\delta u + HOT,
\end{equation}
where $HOT$ refers to ``Higher Order Terms.'' 

Thus, local linearization around an equilibrium point gives a LTI
system. Because you are around an quilibrium point, the system doesn't
want to move.

So the linearized system ends up being
\begin{equation}
  \begin{split}
    \dot{\delta x} &= A \delta x + B \delta u\\
    \delta y &= C\delta x + D \delta u,
  \end{split}
\end{equation}
where
\begin{align}
  A &= \left.\partiald{f}{x}\right|_{(\meq{x},\meq{u}} & 
  B &= \left.\partiald{f}{u}\right|_{(\meq{x},\meq{u}} & 
  C &= \left.\partiald{g}{x}\right|_{(\meq{x},\meq{u}} & 
  D &= \left.\partiald{g}{u}\right|_{(\meq{x},\meq{u}}.
\end{align}

\subsection{Example: Inverted Pendulum}

\begin{figure}
  \centering
  {\Large Insert Digram here}
  \caption{Inverted pendulum}
\end{figure}

\begin{equation}
  \xdot =
  \begin{bmatrix}
    \thetadot \\ \thetaddot
  \end{bmatrix}
  =
  \begin{bmatrix}
    \thetaddot \\ 
    \frac{g}{l}\sin\theta - \frac{b}{ml^2}\thetadot + \frac{u}{ml^2}
  \end{bmatrix}
\end{equation}
Let's linearize this arounf the equilibrium point $ \xdot =
\left[\begin{smallmatrix}
  0 & 0 
\end{smallmatrix}\right]^T$.
Let 
\begin{align}
  \meq{x} &=
\begin{bmatrix}
  0 \\ 0 
\end{bmatrix} & \meq{u} &= 0 & \meq{\theta} &= 0\,(\text{or } \pi)
& \meq{\thetadot} & = 0.
\end{align}
Then we get
\begin{align}
  A &=\left[
  \begin{matrix}
    0 & 1 \\ \frac{g}{l}\cos(\theta) & \frac{-b}{ml^2}
  \end{matrix}\right|_{\meq{x},\meq{u}} &
  B &=
  \begin{bmatrix}
    0 \\ \frac{1}{ml^2}
  \end{bmatrix}&
  C &=\left[
  \begin{matrix}
    2 \\ 2\thetadot
  \end{matrix}\right|_{\meq{x},\meq{u}} &
  D &= \left[ 0 \right]
\end{align}

\section{\emph{Local} Linearization along a trajectory}

\newcommand{\sol}[1]{#1^{sol}}

\begin{figure}
  \centering
  {\Large Insert figure here}
  \caption{A trajectory}
\end{figure}
We want to linearize along each point on the trajectory.
So, given a solution of $\xdot = f(x,u)$,
\begin{equation}
  \begin{split}
    u^{sol} &: [0,\infinity) \rightarrow \mathbb{R}^k\\
    x^{sol} &: [0,\infinity) \rightarrow \mathbb{R}^n\\
    y^{sol} &: [0,\infinity) \rightarrow \mathbb{R}^m\\
  \end{split}
\end{equation}
these give a set of points along the trajectory.  We can write a
system,
\begin{equation}
  \begin{split}
    \dot{\delta x} &= A \delta x + B \delta u\\
    \delta y &= C\delta x + D \delta u,
  \end{split}
\end{equation},
where
\begin{align}
  A &= \left.\partiald{f}{x}\right|_{(\sol{x},\sol{u}} & 
  B &= \left.\partiald{f}{u}\right|_{(\sol{x},\sol{u}} & 
  C &= \left.\partiald{g}{x}\right|_{(\sol{x},\sol{u}} & 
  D &= \left.\partiald{g}{u}\right|_{(\sol{x},\sol{u}}.
\end{align}

\subsection{Example: Satellite}
\begin{figure}
  \centering
  {\Large Insert Figure Here}
  \caption{Satellite Digram}
\end{figure}
Define our states
\begin{align}
  x &=
  \begin{bmatrix}
    r \\ \rdot \\ \theta \\ \thetadot \\ \phi \\ \dot{\phi}
  \end{bmatrix}
  & u &=
  \begin{bmatrix}
    u_r \\ u_\theta \\ u_\phi
  \end{bmatrix}
  & y &=
  \begin{bmatrix}
    r \\ \theta \\ \phi
  \end{bmatrix}
\end{align}
Our equation of motion is given as
\begin{equation}
  \xdot = f(x,u) =
  \begin{bmatrix}
    \rdot \\
    r\thetadot^2\cos^2\phi + r \dot{\phi}^2 - \frac{k}{r^2} +
    \frac{u_r}{m}\\
    \thetadot \\
    -2\frac{\rdot\theta}{r} + 2\thetadot\phidot\tan(\theta) +
      \frac{u_\theta}{mr\cos\phi} \\
    \dot{\phi}\\
    -\thetadot^2\cos\phi \sin\phi - 2\frac{r\dot{\phi}}{r} +
    \frac{u_\phi}{mr} \\
  \end{bmatrix}.
\end{equation}
(in the horizontal plane);
\begin{equation}
  x_0 =
  \begin{bmatrix}
    r_0 & 0 & \omega_0 t & \omega_0 & 0 & 0
  \end{bmatrix}^T
\end{equation}
and 
\begin{equation}
  u_0 =
  \begin{bmatrix}
    0 & 0 & 0
  \end{bmatrix}^T
\end{equation}
We then get the system:
\begin{equation}
  \begin{split}
    \dot{\delta x} &= A \delta x + B \delta u\\
    \delta y &= C\delta x + D \delta u,
  \end{split}
\end{equation},
where
\begin{align}
  A &= \left.\partiald{f}{x}\right|_{(\sol{x},\sol{u}} =
  \begin{bmatrix}
    0& 1& 0& 0& 0& 0\\
    (\dot{\phi}^2\cos^2\phi + \thetadot^2 + 2\frac{k}{r^3} +
    \frac{u}{m} & 0& 0& 2r\thetadot\cos^2\phi & \dots & \dots\\
    \vdots& &\ddots & & & 
  \end{bmatrix}
\\ 
  B &= \left.\partiald{f}{u}\right|_{(\sol{x},\sol{u}} = 
  \begin{bmatrix}
   0 &0&0 \\ 
   \frac{1}{m} &0 &0 \\
   0 &0 &0 \\
   0 &\frac{1}{mr_0} &0 \\
   0 &0 &0 \\
   0 &0 & \frac{1}{mr_0}
  \end{bmatrix} \\
  C &= \left.\partiald{g}{x}\right|_{(\sol{x},\sol{u}} = 
  \begin{bmatrix}
   1 &0 &0 &0 &0 &0 \\
   0 &0 &1 &0 &0 &0 \\
   0 &0 &0 &0 &1 &0 \\
  \end{bmatrix}\\
  D &= \left.\partiald{g}{u}\right|_{(\sol{x},\sol{u}} = 0.
\end{align}

\section{Feedback Linearization}
\begin{equation}
  M(q)\ddot{q}B(q,\qdot) + G(q) = F
\end{equation}
where $F$ is the control input, and $q$ and $\qdot$ can be measured.
Then, 
\begin{equation}
  \ddot{q}M(q)^{-1} \left(- B(q,\qdot) - G(q) + F\right).
\end{equation}
So divide the input control into
\begin{equation}
  F = u_{nl}(q,\qdot) + u_{lin}.
\end{equation}
Then we see the division
\begin{equation}
  M(q)\ddot{q} + \underbrace{B(q,\qdot) + G(q)}_{\text{nonlinear}} =
  u_{nl}(q,\qdot) + u_{lin}. 
\end{equation}
If we let $U_{nl} = B(q,\qdot) +G(q)$, then 
\begin{equation}
\begin{split}
  M(q)\ddot{q}& = u_{lin} \\
  & = -k_p(q-q_d) - k_d(\qdot - \qdot_d).
\end{split}
\end{equation}
We can only cancel, if we know our states and can measure them.
Also, $u_{nl}$, may be large (and possibly not useful; i.e. too large
to actuate or use in a motor, etc.)
\end{document}