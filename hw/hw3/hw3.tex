\documentclass[10pt]{article}
%\input{page}
\usepackage{fullpage}	% Does about the same as the \input{page}...
% ^ part of the preprint package...

\input{symbolmac}
\usepackage{amsmath}
\usepackage{amssymb}  	% provides Rbb, etc.
\usepackage{latexsym}  	% provides \Box
%\usepackage{theorem} \input{proof}
%\pagestyle{empty}
\usepackage{graphicx} 	% for inserting graphics.
%\usepackage{booktabs}	% for fancy tabular environments and stuff
%\usepackage{subfig}	% for subfigures... 
%(This breaks Beamer if the patch isn't applied!!)
%usepackage{hyperref}



%% ******** Set the style: **************
%\lstset{style=matlab}

%\newcommand{\matlab}{{\sc Matlab }} % Or 
\renewcommand{\matlab}{{\sc Matlab }} 

% \newcommand{\homedir}{/home/aepound}
% \newcommand{\figdir}{\homedir/figs}  % picture stuff: ps,fig,tex
%\newtheorem{theorem}{Theorem}
%\newtheorem{corollary}{Corollary}
%\newtheorem{lemma}{Lemma}

\title{Homework 3
}
\author{Andrew Pound}
\date{\today}



\begin{document}
\maketitle

%\begin{abstract}
%\end{abstract}

\section{Problem 1}

Consider a system with input $u$ and output $y$. Three experiments are
performed on the system using the input $u_1(t)$, $u_2(t)$, and
$u_3(t)$ for $t \ge 0$. In each case, the initial state $x(0)$ at $t =
0$ is the same. The corresponding outputs are denoted by $y_1$, $y_2$
and $y_3$. Which one of the following statements are correct if $x(0)
\neq 0$ and which are correct if $x(0) = 0$?
\begin{enumerate}
\item If $u_3 = u_1 + u_2$, then $y_3 = y_1 + y_2$.

\item If $u_3 = 0.5(u_1 + u_2)$ then $y_3 = 0.5(y_1 + y_2)$.
\item If $u_3 = u_1- u_2$ then $y_3 = y_1 - y_2$.
\end{enumerate}
\paragraph{My Answer}
It is unclear what type of system is being used in the question.  
If the system in question is a linear system, then both the zero-state
responses and the zero-input reponse satisfy the superposition
principle.  By this, if the system is linear, that all three ( (a),
(b), and  (c) ) are true both when  $x(0) = 0$. 
But when $x(0) \neq 0$ (i.e. there is some previous state), then the
only one that is correct is (b), because the sum  has twice the
initial state dependence than the single output would have.

%, because then there is
%no zero-input 
%response.

%the initial state 
%which would then be a zero-input response to be added to the
%zero-state reponses that occur from each of the $u_i$ inputs) and they
%are also all true when

Now if the system is \emph{not} linear, then all bets are off.  The
order order interactions may occur in all sort of manners, and the
state may have interesting interactions with the input that cannot
gaurantee that the linearity of the choices is conserved.

%Truthfully, I am confused by this question and don't understand how any
%of the choices are different from each other and how a non-zero
%initial state would determine whether or not superposition would hold
%for a system. I am fairly certain that only the  properties of the
%system itself (i.e. whether it is linear or not) determine whether
%linearity/superposition will be possible.


\section{Problem 2}
 Consider a system described by
 \begin{equation}
   \ddot{y} + 2\dot{y}-3y = \dot{u} -u
 \end{equation}
 \begin{itemize}
 \item Find the transfer function of the system.
 \item So, to find the transfer function, let's first take the Laplace
   transform of the system.
   \begin{equation*}
     Y(s)s^2 + 2Y(s)s - 3Y(s) - Y(0) =
     U(s)s - U(s).
   \end{equation*}
The transfer function is defined with $Y(0) = 0$, so that term will
drop out.  Next we'll gather terms on each side of the equation.
\begin{equation*}
  Y(s)\left(s^2 + s - 3\right) =
  U(s)\left(s - 1\right)
\end{equation*}
Next we'll look at the ratio of output to input to finally give us our
transfer function:
\begin{equation*}
\begin{split}
  \frac{Y(s)}{U(s)} &= \frac{s - 1}{s^2 +
    2s - 3} \\
%  &= \frac{\frac{1}{s}\left(1 - s\right)}{\frac{1}{s^2}\left( 1 + 2s -
%      3s^2\right)} \\
  &=
  \frac{\left(s-1\right)}
       {\left(3s^2-2s-1\right)} \\ 
       &=\frac{(s-1)}{(3s+1)(s-1)}\\
       &= \frac{1}{(3s+1)} = G(s).
\end{split}
\end{equation*}
 \item Find the impulse response of the system.
 \item[] To find the inpulse response of the system, we need to take
   the inverse Laplace transform of the transfer function.
Thus we get:
\begin{equation*}
  \begin{split}
    Y_{imp}(t) &= \mathcal{L}^{-1}\left\{G(s)\right\}\\
    &= \mathcal{L}^{-1}\left\{\frac{1}{3(s+\frac{1}{3})}\right\}\\
    &= \frac{1}{3}\left(e^{-\frac{1}{3}t}u(t)\right) \\
    & = \frac{1}{3}e^{-\frac{1}{3}t}u(t).
  \end{split}
\end{equation*}
 \end{itemize}

\section{Problem 3}
 Show that the following pair of systems are zero-state equivalent, but
not algebraically equivalent.
\begin{itemize}
\item 
  \begin{align*}
    &\qquad \text{A} && \qquad\text{B}\\
    \xdot &=
    \begin{bmatrix}
      1 & 0 \\ 0 1
    \end{bmatrix}x + 
    \begin{bmatrix}
      1 \\ 0
    \end{bmatrix}u
    & \dot{\xbar} &=
    \begin{bmatrix}
      1 & 0 \\ 0 & 2 
    \end{bmatrix} \xbar + 
    \begin{bmatrix}
      1 \\ 0
    \end{bmatrix} u\\
    y &= 
    \begin{bmatrix}
      1 & 0 
    \end{bmatrix}x
    & \ybar &=
    \begin{bmatrix}
      1 & 0 
    \end{bmatrix}\xbar
  \end{align*}

\item[]
In order to show zero-state equivalence, we need to show that bothe
systems realize the same \emph{transfer function}.
\begin{align*}
  G_A &=C(sI-A)^{-1}B + D & G_B &= C(sI - A)^{-1} + D\\
  &=
  \begin{bmatrix}
    1 &0 
  \end{bmatrix}
  \begin{bmatrix}
    s-1 & 0 \\ 0 & s-1
  \end{bmatrix}
  \begin{bmatrix}
    1\\0
  \end{bmatrix}
  &&=
  \begin{bmatrix}
    1 &0 
  \end{bmatrix}
  \begin{bmatrix}
    s-1 & 0 \\ 0 & s-2
  \end{bmatrix}
  \begin{bmatrix}
    1\\0
  \end{bmatrix}\\
  &= \frac{1}{(s-1)^2}
  \begin{bmatrix}
    1 &0 
  \end{bmatrix}
  \begin{bmatrix}
    s-1 & 0 \\ 0 & s-1
  \end{bmatrix}
  \begin{bmatrix}
    1\\0
  \end{bmatrix}
  &&=\frac{1}{(s-1)(s-2)}
  \begin{bmatrix}
    1 &0 
  \end{bmatrix}
  \begin{bmatrix}
    s-2 & 0 \\ 0 & s-1
  \end{bmatrix}
  \begin{bmatrix}
    1\\0
  \end{bmatrix}\\
  &= \frac{1}{(s-1)^2}
  \begin{bmatrix}
    1 &0 
  \end{bmatrix}
  \begin{bmatrix}
    s-1  \\ 0 
  \end{bmatrix}
  &&=\frac{1}{(s-1)(s-2)}
  \begin{bmatrix}
    1 &0 
  \end{bmatrix}
  \begin{bmatrix}
    s-2 \\ 0
  \end{bmatrix}\\
  &=\frac{(s-1)}{(s-1)^2} 
  &&=\frac{(s-2)}{(s-1)(s-2)}\\
  &=\frac{1}{(s-1)}
  &&=\frac{1}{(s-1)}
\end{align*}
Thus, the systems are zero-state equivalent.  Now to show that these
are \emph{not} algebraicly equivalent, wee can take a look at the
characteristic polynomial.
\begin{align*}
  p_A(s) &= (s-1)^2 & P_B&=(s-1)(s-2)
\end{align*}
These are not the same, and thus these two systems are not algebraicly
equivalent. 

\item 
  \begin{align*}
    &\qquad \text{A} && \qquad\text{B}\\
    \xdot &=
    \begin{bmatrix}
      1 & 0 \\ 0 & 1
    \end{bmatrix}x + 
    \begin{bmatrix}
      1 \\ 0
    \end{bmatrix}u
    & \dot{\xbar} &=
      \xbar + u \\
    y &=
    \begin{bmatrix}
      1 & 0
    \end{bmatrix}x
    & \ybar &= \xbar
  \end{align*}

\item[]
Same deal for this one.  Let's look at the transfer functions:
\begin{align*}
  G_A &=C(sI-A)^{-1}B + D & G_B &= C(sI - A)^{-1} + D\\
  &=
  \begin{bmatrix}
    1 &0 
  \end{bmatrix}
  \begin{bmatrix}
    s-1 & 0 \\ 0 & s-1
  \end{bmatrix}^{-1}
  \begin{bmatrix}
    1\\0
  \end{bmatrix}
  &&=
    1 (s-1)^{-1}1\\
  &= \frac{1}{(s-1)^2}
  \begin{bmatrix}
    1 &0 
  \end{bmatrix}
  \begin{bmatrix}
    s-1 & 0 \\ 0 & s-1
  \end{bmatrix}
  \begin{bmatrix}
    1\\0
  \end{bmatrix}
  &&=\frac{1}{(s-1)}\\
  &= \frac{1}{(s-1)^2}
  \begin{bmatrix}
    1 &0 
  \end{bmatrix}
  \begin{bmatrix}
    s-1  \\ 0 
  \end{bmatrix}
  &&\\
  &=\frac{(s-1)}{(s-1)^2} 
  &&\\
  &=\frac{1}{(s-1)}
  &&
\end{align*}
So, we can say that these systems are zero-state equivalent.  Now
let's consider the characteristic polynomials:
\begin{align*}
  p_A(s) &=(s-1)^2 & p_B(s) &= (s-1)
\end{align*}
\end{itemize}

\section{Problem 4}
Consider the following two systems
\begin{align*}
      &\qquad \text{A} && \qquad\text{B}\\
      \xdot &=
      \begin{bmatrix}
        2 & 1 & 2 \\ 0 & 2 & 2\\ 0 & 0 & 1
      \end{bmatrix} x +
      \begin{bmatrix}
        1 \\ 1 \\ 0
      \end{bmatrix}u
      & \dot{\xbar} &=
      \begin{bmatrix}
        2 & 1 & 1 \\ 0 & 2 & 1 \\ 0 & 0 & -1
      \end{bmatrix}\xbar + 
      \begin{bmatrix}
        1 \\ 1 \\ 0
      \end{bmatrix}u\\
      y &=
      \begin{bmatrix}
        1 & -1 & 0
      \end{bmatrix}x 
      & y&=
      \begin{bmatrix}
        1 & -1 & 0
      \end{bmatrix}\xbar
\end{align*}
\begin{itemize}
\item  Are these systems zero-state equivalent?
\item[] 
Once again we need to see if these systems realize the same transfer
function.
So for system A, we get
\begin{equation}
  \begin{split}
    G_A &=
    \begin{bmatrix}
      1 & -1 & 0
    \end{bmatrix}\underbrace{
    \begin{bmatrix}
      s-2 & -1 & -2 \\ 0 & s-2 & -2 \\ 0 & 0 & s-1
    \end{bmatrix}}_{\bar{A}}
    \begin{bmatrix}
      1 \\ 1\\ 0
    \end{bmatrix}\\
    &= \frac{1}{\det(\bar{A})}
    \begin{bmatrix}
      1 & -1 & 0
    \end{bmatrix}
    \begin{bmatrix}
      (s-2)(s-1) & 0 & 0 \\ (s-1) & (s-2)(s-1) & 0 \\ 2+ 2(s-2) &
      2(s-2) & (s-2)^2
    \end{bmatrix}^T
    \begin{bmatrix}
      1 \\ 1 \\ 0
    \end{bmatrix}\\
    &= 
    \begin{bmatrix}
      1 & -1 & 0
    \end{bmatrix}
    \begin{bmatrix}
      \frac{1}{(s-2)} &  \frac{1}{(s-2)^2} &  \frac{2}{(s-2)(s-1)} \\
      0 & \frac{1}{(s-2)} & \frac{2}{(s-2)(s-1)} \\
      0 & 0& (s-2)^2
    \end{bmatrix}
    \begin{bmatrix}
      1 \\ 1 \\ 0
    \end{bmatrix}\\
    &= 
    \begin{bmatrix}
      1 & -1 & 0
    \end{bmatrix}
    \begin{bmatrix}
      \frac{1}{(s-2)} +  \frac{1}{(s-2)^2}\\
       \frac{1}{(s-2)} \\
      0 
    \end{bmatrix}\\
    &=      \frac{1}{(s-2)} +  \frac{1}{(s-2)^2}-
       \frac{1}{(s-2)} \\
    &=  \frac{1}{(s-2)^2}.
  \end{split}
\end{equation}
Now for system B, let's do the same thing.
\begin{equation}
  \begin{split}
    G_B &=
    \begin{bmatrix}
      1 & -1 & 0
    \end{bmatrix}\underbrace{
    \begin{bmatrix}
      s-2 & -1 & -1 \\ 0 & s-2 & -1 \\ 0 & 0 & s+1
    \end{bmatrix}}_{\bar{A}}
    \begin{bmatrix}
      1 \\ 1\\ 0
    \end{bmatrix}\\
    &= \frac{1}{\det(\bar{A})}
    \begin{bmatrix}
      1 & -1 & 0
    \end{bmatrix}
    \begin{bmatrix}
      (s-2)(s+1) & 0 & 0 \\ (s+1) & (s-2)(s+1) & 0 \\ 1+ (s-2) &
      (s-2) & (s-2)^2
    \end{bmatrix}^T
    \begin{bmatrix}
      1 \\ 1 \\ 0
    \end{bmatrix}\\
    &= 
    \begin{bmatrix}
      1 & -1 & 0
    \end{bmatrix}
    \begin{bmatrix}
      \frac{1}{(s-2)} &  \frac{1}{(s-2)^2} &
      \frac{(s-1)}{(s-2)^2(s+1)} \\ 
      0 & \frac{1}{(s-2)} & \frac{1}{(s-2)(s+1)} \\
      0 & 0& (s+1)
    \end{bmatrix}
    \begin{bmatrix}
      1 \\ 1 \\ 0
    \end{bmatrix}\\
    &= 
    \begin{bmatrix}
      1 & -1 & 0
    \end{bmatrix}
    \begin{bmatrix}
      \frac{1}{(s-2)} +  \frac{1}{(s-2)^2}\\
       \frac{1}{(s-2)} \\
      0 
    \end{bmatrix}\\
    &=      \frac{1}{(s-2)} +  \frac{1}{(s-2)^2}-
       \frac{1}{(s-2)} \\
    &=  \frac{1}{(s-2)^2}.
  \end{split}
\end{equation}
Thus, they are zero-state equivalent.
\item  Are they algebraically equivalent?
Let's look at the characteristic polynomials to find this out.
\begin{align*}
  p_A(s) &=(s-2)^2(s-1) &p_B(s) &=(s-2)^2(s+1)
\end{align*}
And hence, they are not algebraicly equivalent.
\end{itemize}

\section{Problem 5}

Find state space realization for the transfer function matrix
\begin{equation}
  \hat{G}(s) =
  \begin{bmatrix}
    \frac{-(12s+6)}{3s+34} & \frac{22s+23}{3s+34}
  \end{bmatrix}
\end{equation}
The first thing that we need to do is to get $\hat{G}_{sp}(s)$ from
$\hat{G}(s)$. We get that by subtracting off the $D$ component.  We
get $D$ by taking the limit
\begin{equation}
\begin{split}
  D &= \lim_{s\rightarrow \infinity} \hat{G}(s) \\
  &= \left.\frac{1}{s + 34/3}
  \begin{bmatrix}
    -4s -2 & 22/3 s + 23/3
  \end{bmatrix}\right|_{s\rightarrow\infinity}\\
  & =
  \begin{bmatrix}
    -4 & 22/3
  \end{bmatrix}
\end{split}
\end{equation}
Now, we can subtract this off of the transfer function to obtain the
strictly proper portion,
\begin{equation}
  \hat{G}(s) =\frac{1}{s + 34/3}
  \begin{bmatrix}
    -4s -\frac{6}{3} + 4s + \frac{136}{3} & \frac{22}{3}s +
    \frac{69}{9} - \frac{22}{3}s - \frac{506}{9}
  \end{bmatrix}
   = \frac{1}{s+34/3}
  \begin{bmatrix}
    \frac{130}{3} & \frac{-437}{9}
  \end{bmatrix}.
\end{equation}
Matching up these terms with the formulas we went over in class, we
can see that our matrices will have the following dimensions
\begin{align}
  A&\in \mathbb{R}^{2\times 2} & B &\in \mathbb{R}^{2\times 2} \\
  C&\in \mathbb{R}^{1\times 2} & D &\in \mathbb{R}^{1\times 2}.
\end{align}
We can identify the terms of $\hat{G}(s)$ as follows
\begin{align}
  \alpha_1 &= 34/3 & N_1 &=
  \begin{bmatrix}
    \frac{130}{3} & \frac{-437}{9}
  \end{bmatrix}.
\end{align}
Using this and equation (4.34) from the book (3rd ed.) we get the
following state space representation in controllable canonical form, 
\begin{align}
  \dot{\xbf} &=
  \begin{bmatrix}
    -\alpha_1 I_p
  \end{bmatrix}\xbf +
  \begin{bmatrix}
    I_p
  \end{bmatrix}\ubf
  & \Rightarrow \hspace{3em}
  \dot{\xbf} &=
  \begin{bmatrix}
    -\frac{34}{3} & 0 \\ 0 & -\frac{34}{3}
  \end{bmatrix}\xbf
   +
   \begin{bmatrix}
     1 & 0 \\ 0 &1
   \end{bmatrix}\ubf\\
  y &=
  \begin{bmatrix}
    N_1
  \end{bmatrix}\xbf + D\ubf
  & y &=
  \begin{bmatrix}
    \frac{130}{3} & -\frac{437}{9}
  \end{bmatrix}\xbf +
  \begin{bmatrix}
    -4 & \frac{22}{3}
  \end{bmatrix}\ubf.
\end{align}
% \bibliographystyle{ieeetr}
% \newcommand{\bibdir}{~}
% \bibliography{\bibdir/~}
\end{document}
