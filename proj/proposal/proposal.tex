\documentclass[12pt, twocolumn]{article}

%\usepackage{fullpage}
\usepackage{setspace}
%\doublespacing

\usepackage{graphicx}
\input{symbolmac}
\usepackage{amsmath}
\usepackage{amssymb}
\usepackage{euscript}
\usepackage{latexsym}
\makeatletter
    \@ifclassloaded{beamer}
    {}% Don't want to use this in the beamer class, either. 
    { %
\usepackage[dvipsnames,usenames]{xcolor}
}
\usepackage{tikz}
\usetikzlibrary{calc}

\makeatother
\usepackage{subcaption}
\usepackage{booktabs}


% I thought about adding this in to be able to switch between lscape
% and pdflscape... but decided I didn't need to...
%\usepackage{ifpdf}
%\ifpdf
%\usepackage{pdflscape}
%\else
\usepackage{lscape}
%\fi

\usepackage{cleveref}

\providecommand{\abs}[1]{\lvert#1\rvert}
\providecommand{\norm}[1]{\lVert#1\rVert}
\definecolor{dark-gray}{gray}{0.2}
\definecolor{light-gray}{gray}{0.3}
\newcommand{\myrule}{\textcolor{light-gray}{\noindent\rule{\linewidth}{0.07mm}}\\}
\newcommand{\editmark}{\begin{center}\textcolor{light-gray}{\noindent\rule{\linewidth}{0.07mm}}\\ 
    To be Edited:\end{center}} 
\renewcommand{\matlab}{{\sc Matlab }} 
\newcommand{\edit}[1]{\editmark \textcolor{dark-gray}{#1}\\\myrule}

\newcommand{\etal}{et~al.~}

\usepackage{comment}


%% ******** For code: ******************
\usepackage{listings}
%\usepackage[usenames,dvipsnames]{color}
%% ******** Define the language: *******
\lstdefinelanguage{matlabfloz}{%
	alsoletter={...},%
	morekeywords={% 						% keywords
	break,case,catch,continue,elseif,else,end,for,function,global,%
	if,otherwise,persistent,return,switch,try,while,...,
        classdef,properties,methods},%
	comment=[l]\%,% 						% comments
	morecomment=[l]...,% 					% comments
	morestring=[m]',%   						% strings 
}[keywords,comments,strings]%
\lstdefinestyle{matlab}{language=matlabfloz,	
		keywordstyle=\color[rgb]{0,0,1},					
		commentstyle=\color[rgb]{0.133,0.545,0.133},	% comments
		stringstyle=\color[rgb]{0.627,0.126,0.941},	% strings
		numbersep=3mm, numbers=left, numberstyle=\tiny% number style
}
%% ******** Set the style: **************
\lstset{style=matlab}

%\newcommand{\matlab}{{\sc Matlab }} % Or 
\renewcommand{\matlab}{{\sc Matlab }} 

\title{ECE 6320 Project Proposal\\
  Dual Axis Control of a Back-Mounted Thrust System
}
\author{Andrew Pound and Dean Lanier}
\date{\today}



\begin{document}
\maketitle

% \begin{abstract}
% The ability to augment human performance with technology can provide
% many benefits.  It has been demonstrated that a booster mounted on the
% back of a person can increase running performance \cite{sugar2014systems}.  This
% project aims to help control the thrusters to minimize thrust in
% directions that would not be beneficial to the user.
% \end{abstract}


\section{Introduction}
% Problem statement
%\section{Background}
% Importance of the problem, existing work, references
The ability to augment human performance with technology can provide
many benefits.  It has been demonstrated that a booster mounted on the
back of a person can increase running performance
\cite{sugar2014systems}.  
Kerestes and Sugar have developed a robotic project that provides
thrust to a runner's hips.  They have produced a working prototype and
have submitted a patent\cite{sugar2014systems}. 
They have demonstrated in clinical trials  that the Jetpack reduces
both time and effort required for a runner to complete a mile run.

A person's body undergoes significant twisting during running.  The
Jetpack 
mounted on the hips of the runner will also experience some of these
same torques, thus diminishing the efficiency of any thrust that
cannot adapt to and correct for these torques.
In order to be able to correct the torsion experienced by the rocket
motors, some sort of vectoring must be possible of the force vector
of the thrust produced. 


\begin{figure}[tbh]
  \centering
  \includegraphics[width=.8\linewidth]{jetpack}
  \caption{A photo of the Jetpack system by Kerestes and Sugar.  (From
    the website of the Human Machine Integration Laboratory)} 
  \label{fig:jetpack}
\end{figure}





\section{Objectives}

This class
project aims to increase the efficiency of how the Jetpack assists the runner.
This will be accomplished by 
modelling a controller that can inform  the rockets to
modify the thrust to minimize the force in 
directions that would not be beneficial to the runner (\emph{i.~e.~}
in directions other than forward).
The idea is that sensors can be placed on the runner's body to provide
a semi-accurate model of how the torso is being twisted.  


     %  \begin{tikzpicture}[domain=0:4]
     %  \draw[very thin,color=gray] (-3.1,-3.1) grid (3.9,3.9);
     %  \draw[->,gray] (-3.2,0) -- (4.2,0) node[right] {$x$};
     %  \draw[->,gray] (0,-3.2) -- (0,4.2) node[above] {$y$};
     %  \draw[-,thick] (1.5,2.5) -- (-1.5,1) -- (-1.5,-2) -- (1.5,-0.5) -- (1.5,2.5);
     %  \draw[-,thick,blue] (2,1.75) -- (-1,.25) -- (-2,-1.75) -- (1,-0.25) -- (2,1.75);
     %  \draw[-,dashed,blue](-1.5,-.75) -- (1.5,.75);
     %  \draw[->,very thick] (0,0) -- (2,-1.25);
     %  \draw[->,very thick,red] (0,0) -- (2,-.8);     
     %  \end{tikzpicture}


\section{Approach}

We want to develop this in an iterative manner.  We hope to first
familiarize ourselves with the mechanics of running, the physics of
how the body twists, and find and understand equations of motion
governing the torso while running.
Then we will model the jetpack as a dual-axis pendulum attached to a
platform that can change its orientation in two independent axes.





%\section{Conclusion}




\bibliographystyle{ieeetr}
\bibliography{references}
% \newcommand{\bibdir}{~}
% \bibliography{\bibdir/~}
\end{document}
